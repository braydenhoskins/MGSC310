\documentclass[]{article}
\usepackage{lmodern}
\usepackage{amssymb,amsmath}
\usepackage{ifxetex,ifluatex}
\usepackage{fixltx2e} % provides \textsubscript
\ifnum 0\ifxetex 1\fi\ifluatex 1\fi=0 % if pdftex
  \usepackage[T1]{fontenc}
  \usepackage[utf8]{inputenc}
\else % if luatex or xelatex
  \ifxetex
    \usepackage{mathspec}
  \else
    \usepackage{fontspec}
  \fi
  \defaultfontfeatures{Ligatures=TeX,Scale=MatchLowercase}
\fi
% use upquote if available, for straight quotes in verbatim environments
\IfFileExists{upquote.sty}{\usepackage{upquote}}{}
% use microtype if available
\IfFileExists{microtype.sty}{%
\usepackage{microtype}
\UseMicrotypeSet[protrusion]{basicmath} % disable protrusion for tt fonts
}{}
\usepackage[margin=1in]{geometry}
\usepackage{hyperref}
\hypersetup{unicode=true,
            pdftitle={Final Report},
            pdfauthor={Noah Estrada-Rand, Brady Hoskins, Charles Filce},
            pdfborder={0 0 0},
            breaklinks=true}
\urlstyle{same}  % don't use monospace font for urls
\usepackage{color}
\usepackage{fancyvrb}
\newcommand{\VerbBar}{|}
\newcommand{\VERB}{\Verb[commandchars=\\\{\}]}
\DefineVerbatimEnvironment{Highlighting}{Verbatim}{commandchars=\\\{\}}
% Add ',fontsize=\small' for more characters per line
\usepackage{framed}
\definecolor{shadecolor}{RGB}{248,248,248}
\newenvironment{Shaded}{\begin{snugshade}}{\end{snugshade}}
\newcommand{\AlertTok}[1]{\textcolor[rgb]{0.94,0.16,0.16}{#1}}
\newcommand{\AnnotationTok}[1]{\textcolor[rgb]{0.56,0.35,0.01}{\textbf{\textit{#1}}}}
\newcommand{\AttributeTok}[1]{\textcolor[rgb]{0.77,0.63,0.00}{#1}}
\newcommand{\BaseNTok}[1]{\textcolor[rgb]{0.00,0.00,0.81}{#1}}
\newcommand{\BuiltInTok}[1]{#1}
\newcommand{\CharTok}[1]{\textcolor[rgb]{0.31,0.60,0.02}{#1}}
\newcommand{\CommentTok}[1]{\textcolor[rgb]{0.56,0.35,0.01}{\textit{#1}}}
\newcommand{\CommentVarTok}[1]{\textcolor[rgb]{0.56,0.35,0.01}{\textbf{\textit{#1}}}}
\newcommand{\ConstantTok}[1]{\textcolor[rgb]{0.00,0.00,0.00}{#1}}
\newcommand{\ControlFlowTok}[1]{\textcolor[rgb]{0.13,0.29,0.53}{\textbf{#1}}}
\newcommand{\DataTypeTok}[1]{\textcolor[rgb]{0.13,0.29,0.53}{#1}}
\newcommand{\DecValTok}[1]{\textcolor[rgb]{0.00,0.00,0.81}{#1}}
\newcommand{\DocumentationTok}[1]{\textcolor[rgb]{0.56,0.35,0.01}{\textbf{\textit{#1}}}}
\newcommand{\ErrorTok}[1]{\textcolor[rgb]{0.64,0.00,0.00}{\textbf{#1}}}
\newcommand{\ExtensionTok}[1]{#1}
\newcommand{\FloatTok}[1]{\textcolor[rgb]{0.00,0.00,0.81}{#1}}
\newcommand{\FunctionTok}[1]{\textcolor[rgb]{0.00,0.00,0.00}{#1}}
\newcommand{\ImportTok}[1]{#1}
\newcommand{\InformationTok}[1]{\textcolor[rgb]{0.56,0.35,0.01}{\textbf{\textit{#1}}}}
\newcommand{\KeywordTok}[1]{\textcolor[rgb]{0.13,0.29,0.53}{\textbf{#1}}}
\newcommand{\NormalTok}[1]{#1}
\newcommand{\OperatorTok}[1]{\textcolor[rgb]{0.81,0.36,0.00}{\textbf{#1}}}
\newcommand{\OtherTok}[1]{\textcolor[rgb]{0.56,0.35,0.01}{#1}}
\newcommand{\PreprocessorTok}[1]{\textcolor[rgb]{0.56,0.35,0.01}{\textit{#1}}}
\newcommand{\RegionMarkerTok}[1]{#1}
\newcommand{\SpecialCharTok}[1]{\textcolor[rgb]{0.00,0.00,0.00}{#1}}
\newcommand{\SpecialStringTok}[1]{\textcolor[rgb]{0.31,0.60,0.02}{#1}}
\newcommand{\StringTok}[1]{\textcolor[rgb]{0.31,0.60,0.02}{#1}}
\newcommand{\VariableTok}[1]{\textcolor[rgb]{0.00,0.00,0.00}{#1}}
\newcommand{\VerbatimStringTok}[1]{\textcolor[rgb]{0.31,0.60,0.02}{#1}}
\newcommand{\WarningTok}[1]{\textcolor[rgb]{0.56,0.35,0.01}{\textbf{\textit{#1}}}}
\usepackage{graphicx,grffile}
\makeatletter
\def\maxwidth{\ifdim\Gin@nat@width>\linewidth\linewidth\else\Gin@nat@width\fi}
\def\maxheight{\ifdim\Gin@nat@height>\textheight\textheight\else\Gin@nat@height\fi}
\makeatother
% Scale images if necessary, so that they will not overflow the page
% margins by default, and it is still possible to overwrite the defaults
% using explicit options in \includegraphics[width, height, ...]{}
\setkeys{Gin}{width=\maxwidth,height=\maxheight,keepaspectratio}
\IfFileExists{parskip.sty}{%
\usepackage{parskip}
}{% else
\setlength{\parindent}{0pt}
\setlength{\parskip}{6pt plus 2pt minus 1pt}
}
\setlength{\emergencystretch}{3em}  % prevent overfull lines
\providecommand{\tightlist}{%
  \setlength{\itemsep}{0pt}\setlength{\parskip}{0pt}}
\setcounter{secnumdepth}{0}
% Redefines (sub)paragraphs to behave more like sections
\ifx\paragraph\undefined\else
\let\oldparagraph\paragraph
\renewcommand{\paragraph}[1]{\oldparagraph{#1}\mbox{}}
\fi
\ifx\subparagraph\undefined\else
\let\oldsubparagraph\subparagraph
\renewcommand{\subparagraph}[1]{\oldsubparagraph{#1}\mbox{}}
\fi

%%% Use protect on footnotes to avoid problems with footnotes in titles
\let\rmarkdownfootnote\footnote%
\def\footnote{\protect\rmarkdownfootnote}

%%% Change title format to be more compact
\usepackage{titling}

% Create subtitle command for use in maketitle
\providecommand{\subtitle}[1]{
  \posttitle{
    \begin{center}\large#1\end{center}
    }
}

\setlength{\droptitle}{-2em}

  \title{Final Report}
    \pretitle{\vspace{\droptitle}\centering\huge}
  \posttitle{\par}
    \author{Noah Estrada-Rand, Brady Hoskins, Charles Filce}
    \preauthor{\centering\large\emph}
  \postauthor{\par}
      \predate{\centering\large\emph}
  \postdate{\par}
    \date{12/12/2019}


\begin{document}
\maketitle

\hypertarget{business-problem}{%
\section{Business Problem}\label{business-problem}}

Our group was interested in finding what characteristics of video games
lead to a successful game. Our results would give developers an insight
into what specific features should be focused on when developing their
games to maximize success. Since there are many different variables
within each game, we looked to find key indicators of a successful game.

\hypertarget{motivation}{%
\section{Motivation}\label{motivation}}

The gaming industry is projected to reach an annual revenue of \$230
billion dollars by the year 2020. Since the gaming industry is quickly
expanding into one of the highest grossing industries in the world, we
wanted to see what types of games are pushing this industry to the top.
When looking at different successful games in our data set, at times
there can be a large disconnect between the resources a game was
developed with and the total number of copies sold. Because of this we
wanted to analyze what specific features led to the less well known
games being considered as successful as the bigger, more popular games.

\hypertarget{summary-statistics}{%
\section{Summary Statistics}\label{summary-statistics}}

The Steam Store dataset consisted of 27,075 observations, or different
games sold on the Steam store, and 18 different features, or
characteristics of the games. This data set was provided by scraping
data from the Steam store by a third party service, Steam Spy. Our data
set consisted of 7 numeric variables: Required Age, Number of
Achievements, Positive Ratings, Negative Ratings, Average Playtime,
Median Playtime, and Price. The dataset also consisted of 11 factored
variables: Application Id, Name, Release Date, English, Developer,
Publisher, Platform, Categories, Genres, Steam Spy set, and Number Of
Owners. We also created 2 factored variables: Simple\_Categories, and
SuccessfulGame. Ultimately we kept only the variables of required\_age,
genres, achievements, average\_playtime, price, simple\_categories, and
successfulGame.

\begin{Shaded}
\begin{Highlighting}[]
\KeywordTok{head}\NormalTok{(steam,}\DecValTok{5}\NormalTok{)}
\end{Highlighting}
\end{Shaded}

\begin{verbatim}
##   required_age genres achievements average_playtime  price
## 1            0 Action            0            17612 9.2032
## 2            0 Action            0              277 5.1072
## 3            0 Action            0              187 5.1072
## 4            0 Action            0              258 5.1072
## 5            0 Action            0              624 5.1072
##   simple_categories successfulGame
## 1      Multi-player              1
## 2      Multi-player              1
## 3      Multi-player              1
## 4      Multi-player              1
## 5     Single-player              1
\end{verbatim}

The drastic decrease in number of variables is due to the fact that many
of the variables proved to be unusable in their current state. Many of
them contained three semicolon delimited values, ultimately making it
difficult to distinguish a discernable category or value for that
particular variable. Moreover, in regards to variables such as publisher
and developer, it was not possible to build meaningful evaluations off
of the data as there were over 10,000 different possible classes. As
such, we used the variables that were still impactful and meaningful to
our analyses. From these remaining variables, the following summary
statistics were calculated

\begin{Shaded}
\begin{Highlighting}[]
\KeywordTok{descr}\NormalTok{(steam)}
\end{Highlighting}
\end{Shaded}

\begin{verbatim}
## Non-numerical variable(s) ignored: required_age, genres, simple_categories
\end{verbatim}

\begin{verbatim}
## Descriptive Statistics  
## steam  
## N: 26876  
## 
##                     achievements   average_playtime      price   successfulGame
## ----------------- -------------- ------------------ ---------- ----------------
##              Mean          45.31             117.46       7.33             0.02
##           Std.Dev         353.96             827.80       7.34             0.14
##               Min           0.00               0.00       0.00             0.00
##                Q1           0.00               0.00       2.16             0.00
##            Median           7.00               0.00       5.11             0.00
##                Q3          23.00               0.00       9.20             0.00
##               Max        9821.00           38805.00      49.91             1.00
##               MAD          10.38               0.00       5.69             0.00
##               IQR          23.00               0.00       7.04             0.00
##                CV           7.81               7.05       1.00             7.04
##          Skewness          13.38              22.16       1.89             6.89
##       SE.Skewness           0.01               0.01       0.01             0.01
##          Kurtosis         189.74             676.18       4.66            45.54
##           N.Valid       26876.00           26876.00   26876.00         26876.00
##         Pct.Valid         100.00             100.00     100.00           100.00
\end{verbatim}

From the table above there are a few significant things to point out.
Our data produced a mean of .02 for the number of successful games. This
is significant because it points out that the overwhelming majority of
the games in this data set we consider to be unsuccessful games. There
were some extreme outliers for achievements and average playtime, so
when we ran our models we removed these observations from the dataset to
prevent skewing our results. We also found it very interesting that the
number of achievements in games was rather low, and it would most likely
drop a significant amount if we had removed the outliers when we
generated this table. The standard deviations for average playtime and
achievements is very high, suggesting that there is a large amount of
variation in our data set.
\includegraphics{FinalReport_files/figure-latex/unnamed-chunk-3-1.pdf}
From the correlation matrix above you can see that there are 2 variables
that are correlated with a successful game: price and average playtime.
This made sense to us for two main reasons: the first is that if a games
price is low or free it is more accessible to the public, and the second
reason is that if a game has a high average playtime there must be a lot
of content in the game that keeps its players for long periods of time.
If the game is more accessible to the public then more players will be
drawn to it, thus increasing popularity and the potential for players to
buy in game cosmetics or expansions. If a game has a high average
playtime, it shows that there is enough content to keep players
interested in the game. To us average playtime provided a lot of
information into a games retention rate of its players, if the retention
rate was higher it is more likely that the game is successful. \#
Summary Plots

\begin{verbatim}
## `geom_smooth()` using method = 'gam' and formula 'y ~ s(x, bs = "cs")'
\end{verbatim}

\includegraphics{FinalReport_files/figure-latex/unnamed-chunk-4-1.pdf}
In looking to find impactful variables to base our analyses on, we
initially believed that more achievements in a game would lead to higher
average playtime. This was not the case, however, as the plot above
clearly shows that some of the games with the lowest achievements have
the highest average playtime.
\includegraphics{FinalReport_files/figure-latex/unnamed-chunk-5-1.pdf}
From the plot above it becomes clear that price does not necessarily
translate to higher average playtime. This is characterized by the equal
dispersion of high average playtimes amongst the different price ranges.
Furthermore, it is intersting to notice that successfulgames are also
equally distributed among the varying price ranges.
\includegraphics{FinalReport_files/figure-latex/unnamed-chunk-6-1.pdf}
As seen in the plot above, it appears that games with essentially no age
requirement have a much broader range of average playtimes than any
other level of required age. This intuitively makes sense because the
more people that can play a game, the more the game will be played, and
thus a higher average playtime overall.
\includegraphics{FinalReport_files/figure-latex/unnamed-chunk-7-1.pdf}
The boxplot above shows the distribution of prices for video games
against the different kinds of genres. While many of the genres have
similar ranges, it is interesting to notice that casual games clearly
have the lowest mean price and range of prices. Moreover, it is
important to notice the ``other'' and ``indie'' categories as having th
widest range of prices, perhaps due to the fact that many of these games
are perhaps created by independent studios, leading them to demand
higher prices as they are unable to produce the games for lower costs.

\hypertarget{models}{%
\section{Models}\label{models}}

In looking to fit models to help solve our business problem, we decided
to use logistic regression, random forest, and tree models. \#\#
Logistic Regreesion

In building our logistic regression model, we first attempted using all
variables as predictors of our successfulGame variable. From this
initial model we were able to see which variables were statistically
significant and proceeded to use these significant variables to build a
new logistic model. The variables used as predictors were price, genres,
required age, and average playtime.

\begin{Shaded}
\begin{Highlighting}[]
\KeywordTok{exp}\NormalTok{(steam_logit}\OperatorTok{$}\NormalTok{coefficients)}
\end{Highlighting}
\end{Shaded}

\begin{verbatim}
##                             (Intercept) 
##                            2.005074e-02 
##                                   price 
##                            1.035616e+00 
##    relevel(genres, ref = "Other")Action 
##                            7.948691e-01 
## relevel(genres, ref = "Other")Adventure 
##                            2.321590e-01 
##    relevel(genres, ref = "Other")Casual 
##                            1.396069e-01 
##     relevel(genres, ref = "Other")Indie 
##                            2.174737e-01 
##   relevel(genres, ref = "Other")Violent 
##                            8.408791e-02 
##                           required_age3 
##                            4.935156e-05 
##                           required_age7 
##                            2.293985e-05 
##                          required_age12 
##                            2.726689e+00 
##                          required_age16 
##                            5.454452e+00 
##                          required_age18 
##                            1.042955e+01 
##                        average_playtime 
##                            1.000477e+00
\end{verbatim}

Looking at price, the model suggests that for every unit increase in
price, a dollar, the likelihood of the game being successful rises by
three percent. Moreover, looking at the different genres of games, it
becomes clear that any game not in the ``Other'' category leads to a
decrease in likelihood of a game being successful. Lastly, when
considering the coefficients for required age, it is interesting to
notice that a game being or required age 12, 16, or 18 leads to a
significant increase in the likelihood of a game being successful.\\
We interpreted the coefficients above as indicating that having a game
considered in the ``Other'' category as well as making the age
requirement higher leads to the largest increases in likelihood of a
game being successful. However, it is very important to take this
interpretation lightly as we must also consider what other variables may
play into this effect. In the case of age requirements, perhaps it is
the more mature content of the video games, and not necessarily the age
requirement itself that drives the increase in likelihood of a game
being successful or not. Moreover, a game being considered in the
``Other'' genre may increase the likelihood of a game succeeding as it
breaks the traditional genre molds of most games, indicative of
innovative and novel game development.
\includegraphics{FinalReport_files/figure-latex/unnamed-chunk-10-1.pdf}
\includegraphics{FinalReport_files/figure-latex/unnamed-chunk-10-2.pdf}
Looking at the ROC plots above, it is clear to see that the optimal
cutoff for our logistic model resides between 0.1 and 0, indicative of a
very low cutoff to maximize true positives and minimize false positives.

\begin{verbatim}
## 
##  
##    Cell Contents
## |-------------------------|
## |                       N |
## |-------------------------|
## 
##  
## Total Observations in Table:  20157 
## 
##  
##                        | steam_train$successfulGame 
## steam_train$pred_class |         0 |         1 | Row Total | 
## -----------------------|-----------|-----------|-----------|
##                      0 |     18916 |       217 |     19133 | 
## -----------------------|-----------|-----------|-----------|
##                      1 |       839 |       185 |      1024 | 
## -----------------------|-----------|-----------|-----------|
##           Column Total |     19755 |       402 |     20157 | 
## -----------------------|-----------|-----------|-----------|
## 
## 
\end{verbatim}

\begin{verbatim}
## 
##  
##    Cell Contents
## |-------------------------|
## |                       N |
## |-------------------------|
## 
##  
## Total Observations in Table:  6719 
## 
##  
##                       | steam_test$successfulGame 
## steam_test$pred_class |         0 |         1 | Row Total | 
## ----------------------|-----------|-----------|-----------|
##                     0 |      6310 |        63 |      6373 | 
## ----------------------|-----------|-----------|-----------|
##                     1 |       279 |        67 |       346 | 
## ----------------------|-----------|-----------|-----------|
##          Column Total |      6589 |       130 |      6719 | 
## ----------------------|-----------|-----------|-----------|
## 
## 
\end{verbatim}

After considering different cutoffs between 0.1 and 0, we decided to use
a cutoff of .04 as going any closer to 0 would increase sensitivity
while sharply penalizing specificity. Although the accuracy, and the
specificity of this model is relatively strong on both test and training
datasets, it becomes clear that there is much to be desired in terms of
sensitivity. With this in mind we set out to improve sensitivity while
maintaining the optimal levels of specificity and accuracy. \#\#
RandomForest

\begin{verbatim}
Next, we began to build a random forest model to hopefully increase the sensitivity of our predictions.  In order to do so, we used cross validation to find the best number of variables to consider at each split of each tree.  We plotted the out of bag error against the number of variables tried and found the optimal number to be 2.  With this  in mind we built our random forest model which yielded the insightful results.
\end{verbatim}

\includegraphics{FinalReport_files/figure-latex/unnamed-chunk-12-1.pdf}
As seen in the importance plot the variables of average playtime and
price are the two most important variables in the 500 trees made in the
random forest model. This largely aligned with the ``random forest
explained'' results that are included outside of this report due to
their formatting. The figures included in that report indicate that
average playtime and required age were used as roots the most while also
having the lowest average depth in each tree, once again highlighting
the most important variables as predictors.

\begin{verbatim}
## 
##  
##    Cell Contents
## |-------------------------|
## |                       N |
## |-------------------------|
## 
##  
## Total Observations in Table:  20157 
## 
##  
##                        | preds_2$preds 
## preds_2$successfulGame |         0 |         1 | Row Total | 
## -----------------------|-----------|-----------|-----------|
##                      0 |     19705 |        50 |     19755 | 
## -----------------------|-----------|-----------|-----------|
##                      1 |       325 |        77 |       402 | 
## -----------------------|-----------|-----------|-----------|
##           Column Total |     20030 |       127 |     20157 | 
## -----------------------|-----------|-----------|-----------|
## 
## 
\end{verbatim}

\begin{verbatim}
## 
##  
##    Cell Contents
## |-------------------------|
## |                       N |
## |-------------------------|
## 
##  
## Total Observations in Table:  6719 
## 
##  
##                             | preds_test_2$preds 
## preds_test_2$successfulGame |         0 |         1 | Row Total | 
## ----------------------------|-----------|-----------|-----------|
##                           0 |      6581 |         8 |      6589 | 
## ----------------------------|-----------|-----------|-----------|
##                           1 |       106 |        24 |       130 | 
## ----------------------------|-----------|-----------|-----------|
##                Column Total |      6687 |        32 |      6719 | 
## ----------------------------|-----------|-----------|-----------|
## 
## 
\end{verbatim}

After making predictions using the random forest model, the above
confusion matrices were produced. Here it is clear that the random
forest model was able to maintain the high level of accuracy and
specificity while also increasing the sensitivity of the model. An
increase of roughly 20-30\% in sensitivity was able to make this model
much more effective at predicting if a game is successful. Thus this
model was able to outperform the initial logistic regression model.

\hypertarget{pruned-tree-model}{%
\subsection{Pruned Tree Model}\label{pruned-tree-model}}

While the random forest model offered us increased performance compared
to the logit model, we still looked for the possibility of a simpler,
yet equally effective model. Thus, we fitted a basic tree model using
all possible variables as predictors. We then pruned the tree to find
the number of splits to give us the lowest amount of deviance from the
truth data, only to find the pruned tree to be identical to the initial
tree.\\
\includegraphics{FinalReport_files/figure-latex/unnamed-chunk-14-1.pdf}
The tree model appeared to draw the same conclusions as the random
forest model, indicating average playtime and price to be the most
important variables to create splits off of.

\begin{verbatim}
## 
##  
##    Cell Contents
## |-------------------------|
## |                       N |
## |-------------------------|
## 
##  
## Total Observations in Table:  20157 
## 
##  
##                                  | basic_preds_train$preds 
## basic_preds_train$successfulGame |         0 |         1 | Row Total | 
## ---------------------------------|-----------|-----------|-----------|
##                                0 |     19699 |        56 |     19755 | 
## ---------------------------------|-----------|-----------|-----------|
##                                1 |       337 |        65 |       402 | 
## ---------------------------------|-----------|-----------|-----------|
##                     Column Total |     20036 |       121 |     20157 | 
## ---------------------------------|-----------|-----------|-----------|
## 
## 
\end{verbatim}

\begin{verbatim}
## 
##  
##    Cell Contents
## |-------------------------|
## |                       N |
## |-------------------------|
## 
##  
## Total Observations in Table:  6719 
## 
##  
##                                 | basic_preds_test$preds 
## basic_preds_test$successfulGame |         0 |         1 | Row Total | 
## --------------------------------|-----------|-----------|-----------|
##                               0 |      6569 |        20 |      6589 | 
## --------------------------------|-----------|-----------|-----------|
##                               1 |       115 |        15 |       130 | 
## --------------------------------|-----------|-----------|-----------|
##                    Column Total |      6684 |        35 |      6719 | 
## --------------------------------|-----------|-----------|-----------|
## 
## 
\end{verbatim}

From the tree model, we once again made predictions using the model to
observe its performance against the truth data. Ultimately, we found the
tree model to perform similarly to the logistic model in regards to
sensitivity, whilst maintaining high accuracy and specificity, just as
the other two models had. Thus, while this model was able to offer a
human readable model, it offered lower sensitivity than the random
forest model.

\hypertarget{conclusions}{%
\section{Conclusions}\label{conclusions}}


\end{document}
